% !TeX root = ../documentation.tex
% !TeX spellcheck = de_DE


\documentclass[chaostheorie/chaostheorie.tex]{subfiles}

\begin{document}
	\section{Chaostheorie}
	Das Wetter ist ein chaotisches System. Dies hat auch Herr Lorenz schnell erkannt. Er folgerte daraus, dass er ein Modell erstellen muss, dass selber chaotisches Verhalten zeigen muss. Doch was ist chaotisches Verhalten denn?
	
	Diese Modelle werden chaotisch genannt, weil sie auf den ersten Blick keiner Gesetzmässigkeiten folgen. Dennoch haben diese Systeme bei genauer Untersuchung sehrwohl wiederkehrende Verhaltensweisen. Zum Beispiel ähneln Teile der Plotten einander, sie sind Fraktale oder zeigen ähnlichen Plotten. Die ersten zwei Phenomene können bei dem Mandelbrot gezeigt werden \cite{Gleick} und die Plotten von den Lorenz Gleichungen eingebaut.
	
	Im Generellen ist die Chaostheorie ein Bereich der Mathematik der sich mit nicht linearen dynamischen Systemen auseinander setzt.
	
	\paragraph{Dynamik} beschreibt Systeme deren Variablen sich über die Zeit verändern. Bei jedem Zeitpunkt besitzt ein dynamisches System ein Vektor der auf einen Punkt im Ergebnisraum zeigt. Dieser Vektor wird dann als Basis für die Berechnung des nächsten Punktes im Koordinatensystem verwendet.
	
	Vielfach sind dynamische Systeme auch deterministisch \cite{https://en.wikipedia.org/wiki/Dynamical_system}. Das bedeutet, dass aus dem jetztigen Ausgangspunkt den nächsten Punkt im System berechnet werden kann.
	
	% Iterative Berechnungsweise von Dynamischen Systemen
	
	\paragraph{Nicht-lineare Systeme} % Viele sind Oszillatoren mit mehreren Variablen und Rückkopplungen
	Bei linearen Systemen kann eine Proportionalität zwischen der Veränderung der Eingangsgrössen und den Ausgangsgrössen festgestellt werden. Währendem bei den nicht-linearen Systemen genau diese Eigenschaft nicht vorhanden ist.
	
	Nicht-lineare Systeme besitzen Rückkopplungen von verschiedenen Ausgangsgrössen auf die Eingangsgrössen. Wichtig zum verstehen ist hier die Mehrzahl, um nicht-linear zu sein müssen mehrere Rückkopplungen vorhanden sein. Bei einer Rückkopplung ist es sehr wahrscheinlich, dass das System sich linear verhaltet.
	
	Mathematiker beschreiben solche Systeme gerne mit Gleichungssytemen.
	
	\paragraph{Chaostheorie} ist ein Teilbereich der oben erwähnten dynamischen Systeme. Ich werde im Folgenden Systeme und Modelle als Synonym betrachten. Chaotische Systeme besitzen ganz wenige Eingangsparameter.
	
	Die entscheidende Eigenschaft nach welcher diese Systeme kategoriesiert werden ist ihre Empfindlichkeit auf die Eingangsparameter. Eine kleine Änderung in den Eingangsparamter kann grosse Änderung der Werte auslösen.
	
	Es gibt aber auch andere äussere Einwirkungen, die eine Veränderung der Werte auslösen kann. Zum Beispiel kann der Rundungsfehler, welcher bei modernen CPU‘s nicht-vorhersagbar ist, die Ergebnisse stark verändern. Weil jeder Wert auf den vorherigen Werten aufbaut wird der Rundungsfehler die Werte immer stärker verfälschen.
	
	\paragraph{Manifold}
	
	\paragraph{Butterfly Effect}
	
\end{document}