\documentclass[12pt,a4paper]{article}
\usepackage[utf8]{inputenc}
\usepackage{amsmath}
\usepackage{amsfonts}
\usepackage{amssymb}
\usepackage{graphicx}
\usepackage{float}
\usepackage{hyperref}\hypersetup{%
	pdfborder = {0 0 0}
}
\usepackage{listings}
\usepackage{xcolor}


\renewcommand{\familydefault}{\sfdefault}
\newcommand{\sectionbreak}{\clearpage}


\definecolor{OliveGreen}{cmyk}{0.64,0,0.95,0.40}
\definecolor{lightlightgray}{gray}{0.9}




\author{Oli Dias}
\title{Meteorologie - Summary}
\begin{document}
\maketitle
\newpage
\section{Lorenz Attraktor}
Grundsätzlich: Visualisieren des Lorenz-Attraktors und mathematisches Beschreiben dieses Phänomens. Idealerweise finden von Sattel-Knoten-Bifurkationen. Im Skript p.43-46 Informationen zum Lorenz-Attraktor.\\

Ideen zum Vorgehen:
\begin{itemize}
	\item Visualisierung mit Abhängigkeiten zu den physikalischen Grössen (Temperaturänderungen, Viskosität, etc), die die Simulation verändern
	\item Finden von Sattel-Knoten Bifurkation und eventuell Darstellen
\end{itemize}

\end{document}