% !TeX root = presentation.tex
% !TeX spellcheck = de_DE

\documentclass[presentation.tex]{subfiles}

\begin{document}
	\begin{frame}{Physik}{Energieerhaltungssätze}
		\begin{columns}[c]
			\begin{column}{.3\textwidth}
				Dichte
			\end{column}
			\begin{column}{.3\textwidth}
				Geschwindigkeit
			\end{column}
			\begin{column}{0.3\textwidth}
				Temperatur
			\end{column}
		\end{columns}
	\end{frame}

	\begin{frame}{Physik}{Einwirkungen auf Zelle}
		\begin{columns}[c]
			\begin{column}{.3\textwidth}
				\begin{block}{Primär}
					Dichte und Temperatur
				\end{block}
			\end{column}
			\begin{column}{.3\textwidth}
				\begin{block}{Sekundär}
					Geschwindigkeit
				\end{block}
			\end{column}
			\begin{column}{0.3\textwidth}
				\begin{block}{Tertiär}
					Temperatur
				\end{block}
			\end{column}
		\end{columns}
	\end{frame}

	\begin{frame}{Physik}{Gleichungen des Lorenzsystemes}
		\begin{columns}[c]
			\begin{column}{.5\textwidth}
				\begin{block}{Kontinuitätsgleichung}
				\end{block}
				\begin{block}{Navier-Stokes Gleichung}
				\end{block}
				\begin{block}{Temperaturgleichung}
				\end{block}
			\end{column}
			\begin{column}{.5\textwidth}
				\begin{align*}
				\dot{x} &= -\sigma x + \sigma y\\
				\dot{y} &= \rho x - <- xz\\
				\dot{z} &= -\beta z + xy\\
				\end{align*}
			\end{column}
		\end{columns}
	\end{frame}
\end{document}